
\usepackage[T1]{fontenc}

\usepackage{fontawesome}
\usepackage[hidelinks,
    colorlinks,
    filecolor={blue},
    urlcolor={blue}]{hyperref}
\setlength{\parindent}{0cm}
\usepackage{paracol}
\usepackage{ifthen}

\usepackage{tikz}
\usepackage{tikz-3dplot}
\usepackage{smartdiagram}
\usepackage{float}

\usepackage{array}
%\setlength\extrarowheight{1.5em} % adds some spacing between the lines of the CV's tabulars
% not used here as not to interfer with other tables
\usetikzlibrary{decorations.text}
\usetikzlibrary{fadings}
\usetikzlibrary{calc}

\usetikzlibrary{shapes.misc,positioning}
\usetikzlibrary{arrows}
\usetikzlibrary{arrows.meta}
\usetikzlibrary{backgrounds} 
\usetikzlibrary{shadings}
\usetikzlibrary{calendar} 
\usetikzlibrary{er}
\usetikzlibrary{patterns} % horizontal lines, vertical lines, north east lines, north west lines, grid, crosshatch, dots, crosshatch dots, fivepointed stars, sixpointed stars, bricks
\usetikzlibrary{shapes} 
\usetikzlibrary{shapes.geometric}
\usetikzlibrary{decorations}
\usetikzlibrary{topaths}

\usepackage{graphicx}
%\usepackage[usenames, dvipsnames]{color} 
% https://de.sharelatex.com/learn/Using_colours_in_LaTeX#!#Reference_guide z.B. \color{RubineRed}
%\usepackage{xcolor}


% avoid line overflow
\setlength{\emergencystretch}{2pt}
%---------------------------------------------------------------------------------

\usepackage{titlesec} % Allows creating custom \sections

% Format of the section titles
\titleformat{\section}{
\scshape\Large\raggedright}{}{0em}{}[\titlerule] % smallcaps, Large, continuous line - looks better if two columns, might look a bit too dramatic if just one ;) 
\titlespacing{\section}{0pt}{12pt}{5pt} % Spacing around titles {<left spacing>}{<before spacing>}{<after spacing>}
%----------------------------------------------------

 \newcommand{\cvkeyword}[3]{
 \colorbox{#2}{\textcolor{#3}{#1}} \phantom{}%
 }
 
%------------------------
\newcommand{\cvtag}[1]{% from alta-cv: https://www.overleaf.com/latex/templates/altacv-template/trgqjpwnmtgv
  \tikz[baseline]\node[draw=black!40,rounded corners,inner sep=0.4em]{\color{black!50}#1};
}


%------------------- ICON Cross

\newcommand{\iconcross}[7]{%
\begin{tikzpicture}%
\draw[ultra thick,draw=#2] (-2,-2) -- (2,2);%
\draw[ultra thick, draw=#2] (-2,2) -- (2,-2);%
\node[above=0.5em,text=#3] at (0,1) {#1 #4};%
\node[below=0.5em,text=#3] at (0,-1) {#1 #5};%
\node[right=0.5em,text=#3] at (1,0) {#1 #6};%
\node[left=0.5em,text=#3] at (-1,0) {#1 #7};%
\end{tikzpicture}%
}
 

\usepackage{smartdiagram}
%-----------------------------------------------------
% set smartdiagram colours, from: https://github.com/Johayon/Data-Scientist-Resume-LaTeX/blob/master/twentysecondcv.cls


%-------------------------------------------------------
\smartdiagramset{
    bubble center node font = \footnotesize,
    bubble node font = \footnotesize,
    % specifies the minimum size of the bubble center node
    bubble center node size = 0.5cm,
    %  specifies the minimum size of the bubbles
    bubble node size = 0.5cm,
    % specifies which is the distance among the bubble center node and the other bubbles
    distance center/other bubbles = 0.3cm,
    % sets the distance from the text to the border of the bubble center node
    distance text center bubble = 0.5cm,
    % set center bubble color
    bubble center node color = pblue,
    % define the list of colors usable in the diagram
    set color list = {materialcyan, orange, green, materialorange, materialteal, materialamber, materialindigo, materialgreen, materiallime},
    % sets the opacity at which the bubbles are shown
    bubble fill opacity = 0.6,
    % sets the opacity at which the bubble text is shown
    bubble text opacity = 1,
    description title text width=0.5cm,
    description title width=0.5cm,
    description width=5cm,
    description text width=5cm,
    descriptive items y sep =1.25,
    back arrow distance = 0,
    back arrow disabled = true,
    border color = white
}
%-------------------------------
\newcommand{\lorem}{Lorem ipsum dolor sit amet, consectetur adipiscing elit. Donec a diam lectus.} % Dummy text

%--------------------------------------------------- pictures
\newcommand{\roundpic}[1]{\begin{figure}[H]\tikz  \draw [path picture={ \node at (path picture bounding box.center){\includegraphics[height=3.4cm]{#1}} ;}] (0,2) circle (1.7) ;\end{figure}}

\newcommand{\squarepic}[1]{\begin{figure}[H]\tikz  \draw [path picture={ \node at (path picture bounding box.center){\includegraphics[height=4cm]{#1}} ;}] (0,0) - - (-2,2) - - (0,4) - - (2,2) - - cycle ;\end{figure}}


%------------------------------------ pictoFraction
\newcommand{\icon}[3]{\phantom{x}{#3\color{#2}#1}\phantom{x}}
%------------------- pictogram Fraction: pictoFraction
\newcommand{\pictofraction}[6]{%
\pgfmathparse{#3 - 1}\foreach \n in {0,...,\pgfmathresult}{\icon{#1}{#2}{#6}}%
\pgfmathparse{#5 - 1}\foreach \n in {0,...,\pgfmathresult}{\icon{#1}{#4}{#6}}%
}



%----------------------------------------------------- font highlighting / boxes with background color 
\newcommand{\bg}[3]{\colorbox{#1}{\bfseries\color{#2}#3}}
\newcommand{\bgupper}[3]{\colorbox{#1}{\color{#2}\huge\bfseries\MakeUppercase{#3}}}

%------------------------- Bubble Diagram
\newcommand{\bubblediagram}[1]{\smartdiagram[bubble diagram]{#1}}


\newcommand{\skillbubble}[2]{%
\begin{tikzpicture}%
\draw[draw=none,fill=#1] (0,0) circle (0.#2);%
\end{tikzpicture}%
}

\newcommand{\hobbyicon}[5]{%
\begin{tikzpicture}%
\draw[draw=none,fill=#3] (0,0) circle (0.5);%
\node[](icon) at (0,0) {#4#1};%
\node[below=#5,align=center] at (icon) {#2};%
\end{tikzpicture}
}

\renewcommand*{\emailsymbol}{{\small\faEnvelopeO}~}  


%-------------------------------------------- rules / separators

\newcommand{\dashrule}[2]{\begin{figure}[H]\begin{minipage}[t]{#1\textwidth}\tikz \draw[loosely dashed,#2] (0pt,0pt) -- (\textwidth,0pt);\end{minipage}\end{figure}}

\newcommand{\dotrule}[2]{\begin{figure}[H]\begin{minipage}[t]{#1\textwidth}\tikz \draw[loosely dotted, ultra thick,#2] (0,0) -- (\textwidth,0);\end{minipage}\end{figure}}


%-------------------------------------------------------




\newcommand{\event}[5]{%
{%
\draw[draw=black, line width=0.2em,anchor=west] (0,#1) -- (#5,#1);%
\node[fill=#2,right,inner sep=0.5em] 	at (#5,#1) {\textbf{#3}~~#4 };%
} % startpunkt #1. Farbe #2, Detaildatum #3, Beschreibung #4, Entfernung von Timeline #5
}

%---------------------------------------------------------
\newcommand{\barrule}[3]{\hspace{0.5em}
{\color{#3}\rule[\baselineskip]{#1\textwidth}{#2}}\vspace{0.5em}
}

\newcommand{\cvevent}[6]{{#1}&\textbf{#2}\newline\textsc{#3} $\cdot$ {#4 ~\faMapMarker}\newline{\color{black!70}\footnotesize #5}\vspace{1.5em} & \raisebox{-0.7\height}{\includegraphics[height=1cm]{#6}}}

\newcommand{\cvdegree}[6]{{#1} & \textbf{#2}\newline\textsc{#3} $\cdot$ {#4 {\phantom{i}\footnotesize ~\faUniversity}}\newline{\color{black!70}\scriptsize #5}\vspace{1.5em} & \raisebox{-0.7\height}{\includegraphics[height=1cm]{#6}}}








% -----------------------------------------------------------------------------------

\newcommand{\simpleheader}[5]{
\tikz[remember picture,overlay] {%
\node[rectangle, fill=#1, anchor=north, minimum width=\paperwidth, minimum height=2.8 cm](header) at (current page.north){};%
\node[draw=none, align=left](name) at (header) {%
{\Huge \color{#5} #2 \textbf{#3} }%
};%
\node[draw=none, below](description) at (name.south) {\color{white}#4};% 
}\vspace{-0.7cm}%
}


\newcommand{\infobubble}[4]{
\scalebox{1.3}{
\begin{tikzpicture}
\draw[draw=#2,fill=#2] (0,0) circle (0.2cm);
\node[] at (0,0) {\color{#3}\textbf{#1}};
\node[right=0.2cm] at (0,0) {\texttt{#4}};
\end{tikzpicture}
}
}
